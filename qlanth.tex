\documentclass{article}
\usepackage[utf8]{inputenc}
\usepackage[paperwidth=8.5in,paperheight=11in]{geometry}
\geometry{left=2cm, right=2cm, top=2cm, bottom=2cm}
% Include the listings package for syntax highlighting
\usepackage{listingsutf8}
% Optional: include the xcolor package for more advanced color definitions
\usepackage{xcolor}
\usepackage{multicol}
\usepackage{amssymb,
			amsmath}
\usepackage{qlanth} 

% Listing configuration
\lstset{
    language=Mathematica,                  % Set your language (you can change the language for each code-block optionally)
    basicstyle=\ttfamily\small,       % The size of the fonts that are used for the code
    literate=
        {\\[Alpha]}{{$\alpha$}}{1}
        {\\[Beta]}{{$\beta$}}{1}
        {\\[Gamma]}{{$\gamma$}}{1}
        {\\[Zeta]}{{$\zeta$}}{1}
        {\\[VerticalSeparator]}{|}{1}
        ,  
    keywordstyle=\color{blue},        % Keyword style
    stringstyle=\color{red},          % String literal style
    commentstyle=\color{cyan},       % Comment style
    morecomment=[l][\color{magenta}]{\#},
    breakatwhitespace=false,          % Sets if automatic breaks should only happen at whitespace
    breaklines=true,                  % Sets automatic line breaking
    captionpos=b,                     % Sets the caption-position to bottom
    keepspaces=true,                  % Keeps spaces in text, useful for keeping indentation of code (possibly needs columns=flexible)
    showspaces=false,                 % Show spaces everywhere adding particular underscores; it overrides 'showstringspaces'
    showstringspaces=false,           % Underline spaces within strings only
    showtabs=false,                   % Show tabs within strings adding particular underscores
    tabsize=2,                        % Sets default tabsize to 2 spaces
    frame=single,                     % Adds a frame around the code
    numbers=left,                     % Where to put the line-numbers; possible values are (none, left, right)
    numberstyle=\tiny\color{gray},    % Style used for line-numbers
    stepnumber=1,                     % Step between two line-numbers. If it's 1, each line will be numbered
    numbersep=5pt,                    % How far the line-numbers are from the code
    xleftmargin=0.5cm,                % Margin from left
    xrightmargin=0.5cm              % Margin from right
}

% \lstset{
%     language=Mathematica,                  % Set your language (you can change the language for each code-block optionally)
%     basicstyle=\ttfamily\small,       % The size of the fonts that are used for the code
%     literate= {BeginPackage}{{$\alpha$}}{1}            % Margin from right
% }

\begin{document}

\title{qlanth}
\author{}
\date{}
\maketitle

\section{Notation}

\begin{gather}
    \overbracket{\Lambda}^{\text{Shorthand for all other quantum numbers}} \\ 
    \overbracket{\lorb}^{\text{orbital angular momentum}} \\
    \overbracket{\redbraopket{\Lambda LS}{\op{O}}{\Lambda' L'S'}}^{\text{LS-reduced matrix element of operator }\hat{O}\text{ between }{\Lambda LS} \text{ and } {\Lambda' L'S'}} \\
    \overbracket{\redbraopket{\Lambda LSJ}{\op{O}}{\Lambda' L'S'J'}}^{\text{LSJ-reduced matrix element of operator }\hat{O}\text{ between }{\Lambda LSJ} \text{ and } {\Lambda' L'S'J'}} \\
    \overbracket{\op{X}^{(k)}}^{\text{spherical tensor operator of rank k}} \\
    \overbracket{\LSterm{2 S + 1}{\alpha{L}}\equiv \ket{\alpha{LS}}}^{\text{Spectroscopic term } {\alpha{LS}} \text{ in Russel-Saunders notation }} \\
    \overbracket{\op{X}_q^{(k)}}^{\text{q-component of the spherical tensor operator }\op{X}^{(k)}} \\
    \overbracket{\cfp{\lorb^{n-1}\alpha'L'S'}{\lorb^n\alpha{L}{S}}}^{\text{The coefficient of fractional parentage from the parent term }\ket{\lorb^{n-1}\alpha'{L'S'}}\text{ for the daughter term }\ket{\lorb^{n}\alpha{LS}}}  
\end{gather}

\section{Definitions}

\begin{gather}
    \overbracket{\op{u}^{(k)}}^{\text{irreducible unit tensor operator of rank k}} \\
    \overbracket{\op{U}^{(k)} \DEF \sum_{i=1}^{n} \op{u}^{k} }^{\text{symmetric unit tensor operator for n equivalent electrons}} \\
    \overbracket{\cfp{\lorb^{n-1}\alpha'L'S'}{\lorb^n\alpha{L}{S}}}^{\text{The coefficient of fractional parentage from the parent term }\ket{\lorb^{n-1}\alpha'{L'S'}}\text{ for the daughter term }\ket{\lorb^{n}\alpha{LS}}}
\end{gather}

\section{The Effective Hamiltonian} 

\begin{multicols}{2}

Electrons in a multi-electron ion are subject to a number of interactions. They are subject to the attraction towards the nucleus around which they orbit. They are subject to the repulsion that they experience from other electrons. They have spin, also, so they are also subject to a number of magnetic interactions. The spin of each electron interacts with the magnetic field generated by either its own orbital angular momentum, or the orbital angular momentum of another electron. Finally, between pair so felectrons, the spin of one of them will also have an influence the other through the interaction of their respective magnetic dipoles.

This is already a good number of terms to consider in the description of a free ion. However, if we want to take into account the possibility that this description may also hold good for an ion inside a crystal, then we need to add elements to this description that are due to the crystal. The simplest way in which this effect is often included is through the so called crystal-field, which more accurately is often understood as originating from the electric field that an ion feels from the surrounding charges in the crystal lattice.

The Hilbert space of a multi-electron ion is a large auditorium. In principle the Hilbert space should have a countable infinity of discrete states and a uncountable infinity of states to describe the unbound states. This is clearly too much to handle, but thankfully, this large stage can be put in some order thanks to the exclusion principle. The exclusion principle (together with that graceful tendency of things to drift downwards the energetic wells) provides the shell structure. This shell structure, in turn, makes it possible that an atom with many electrons, can be effectively be described as an aggregate of an inert core and a fewer active valence electrons.  

Take for instance a triply ionized neodymium atom. In principle, this gives us the daunting task of dealing with 57 electrons. However, 54 of them arrange themselves in a xenon core, so that we are only left to deal with only three. Three are still a challenging task, but much less so than fifty seven. Furthermore, the exclusion principle also guides us in what type of orbital we could possibly place these three electrons, in the case of the lanthanide ions, this being the 4f orbitals. But not really, there are many more unoccupied orbitals outside of the xenon core, two of these electrons, if they are willing to pay the energetic price, they could find themselves in a 5d or a 6s orbital.  

Here we shall assume a single-configuration description. Meaning that all the valence electrons in the ions that we study here will all be considered to be located in f-orbitals, or what is the same, that they are described by \fn wavefunctions. This is, however, a harsh approximation, but thankfully one can make some amends to it. The terms that arise in the single configuration description because of omitting all the other possible orbitals where the electrons might find themselves, this is what we call \textit{configuration interaction}. 

These effects can be brought within the simplified description only through the help of perturbation theory. The task not the usual one of correcting for the energies/eigenvectors given an added perturbation, but rather to consider the effects of using a truncated Hilbert space due to a known interaction. What results from this is are operator that now act solely within the single configuration but with a convoluted coefficient that depends on overlaps between different configurations. This  coefficient one could try to evaluate, and there are some that have trodden this road. Others simply label that complex expression with an unassuming symbol, and leave it as a parameter that one can fit hopes to fit against experimental data. It is from this that the parameters $\alpha, \beta, \gamma, P^0, P^2, \text{and } P^4$ enter into the description that we shall use here. 

Something that is also borne out of the configuration interaction analysis is that their influence also modifies previously present intra-configuration operators. For instance, part of the configuration interaction influence that results from the Coulomb repulsion between electrons brings about new operators that need to be included, but they also contribute to the intra-configuration Slater integrals. As such, every parameter in the Hamiltonian becomes a quantity to be fitted against spectroscopic data.

When finding the matrix elements of the Hamiltonian defined by these terms, one also requires the specification of the basis in which the matrix elements will be computed. What we shall use here are states determined by five quantum numbers: the total orbital angular momentum $L$, the total spin angular momentum $S$, the total angular momentum $J$, and the projection of the total angular momentum along the z-axis $M_J$. To account for the fact that there might be a few different ways to amount for a given LS, it becomes necessary to have a fifth quantum number that discriminates between these different cases. This other quantum number we shall simply call $\alpha$, which in the notation of Nielson and Koster is simply an integer number that enumerates all the possible LS in a given \fn configuration.

\end{multicols}

Putting all of this together leads to the following Hamiltonian. In there, ``v-electrons'' is shorthand for valence electrons.

\begin{align}
	\ham &= \underbracket{\hamKineticSymbol}_{\text{kinetic}}
		 + \underbracket{\hamNuclearCoulombSymbol}_{\text{e:shielded nuc}}
		 + \underbracket{\hamCoulombEESymbol}_{\text{e:e}}
		 + \underbracket{\hamSpinOrbitSymbol}_{\text{spin-orbit}}
		 + \underbracket{\hamCrystalFieldSymbol}_{\text{crystal field}}
		 + \underbracket{\hamSpinSpinSpinOtherOrbitSymbol}_{
		 			\substack{
		 				\text{spin:spin} \\ 
		 				\text{and spin:other-orbit}
		 				}
		 			} \\
		 & + \underbracket{\hamTreesSymbol}_\text{Trees effective op} 
		 + \underbracket{\hamGTwoSymbol}_\text{G${}_2$ effective op} 
		 + \underbracket{\hamSOSevenSymbol}_\text{$\SO{7}$ effective op} 
		 + \underbracket{\ham_\text{f3}}_{\substack{
		 \text{effective} \\
		 \text{three-body}}} 
		 + \underbracket{\hamECSOSymbol}_{
		 		\substack{
		 			\text{electrostatically} \\
		 			\text{correlated spin:orbit}
		 			}
		 			} \\
	\hamKineticSymbol &= -\frac{\hbar^2}{2m_e}\sum_{i=1}^N \nabla_i^2 \text{ (kinetic energy of N v-electrons)}\\
	\hamNuclearCoulombSymbol &= \hamNuclearCoulomb \text{ (interaction of v-electrons with shielded nuclear charge)} \\
	\ham_\text{e:e} &= \hamCoulombEE \text{ (v-electron:v-electron repulsion)} \\
	\hamSpinOrbitSymbol &= \begin{cases} 
			\hamSpinOrbit \text{ with } \xi{(r_i)} = \frac{\hbar^2}{2 m^2 c^2 r_i} \frac{\diff{V_\text{sn}(r_i)}}{\diff{r_i}} \\
			\sum_{i=1}^N \zeta \paren{\op{\vec{s}_i} \cdot \op{\vec{l}_i}} {{\substack{
						\text{ with $\zeta$ the radial average of $\xi(r_i)$} \\ 
						\text{ or used as phenomenological parameter}
						}
					}}
			\end{cases} \\
	\hamCrystalFieldSymbol &= \hamCrystalField {
		\substack{
			\text{ (crystal field interaction of v-electrons with} \\
			\text{electrostatic field due to surroundings)}
			}
			}\\
	\hamSpinSpinSpinOtherOrbitSymbol &= \hamSpinSpinSpinOtherOrbit \\
	\casimir{\anyGroup} &\DEF \text{The Casimir operator of group $\anyGroup$.}\\
	\hamTreesSymbol &= \alpha\casimir{\mathbb{R}^3} = \alpha \op{L}^2 = \underbrace{\alpha L (L+1)}_\text{in LS coupling} \text{ (Trees effective operator\footnotemark)} \\
	\hamGTwoSymbol      &= \beta\casimir{\text{G}_2} \\
	\hamSOSevenSymbol   &= \gamma\casimir{\SO{7}} \\
	\ham_\text{f3} &= T'^{2}t'_2 + \hamEffectiveThreeBody \text{ (effective three-body operators $\op{t_i}$ with strengths $T_i$)\footnotemark} \\
	\hamECSOSymbol &= \hamECSO
\end{align}

\subsection{$\hamKineticSymbol$: kinetic energy}

\begin{equation}
    \hamKineticSymbol = -\frac{\hbar^2}{2m_e}\sum_{i=1}^N \nabla_i^2 \text{ (kinetic energy of N v-electrons)}
\end{equation}

Within the basis that we'll use, the kinetic energy simply contributes a constant energy shift, and since all we care about are energy transitions, then this term can be omitted from the analysis.

\subsection{$\hamNuclearCoulombSymbol$: e:shielded nuc}

\begin{equation}
\hamKineticSymbol = -\frac{\hbar^2}{2m_e}\sum_{i=1}^N \nabla_i^2 \text{ (kinetic energy of N v-electrons)}
\end{equation}

Instead of using the shielded nuclear charge this could have been instead the bare nuclear charge, but then we would have needed to take into account the repulsion from the electrons in closed shells. Here we are already bringing some simplification in that we approximate the compound effect on the valence electrons due to the charge of the filled shells and the charge of the nucleus is that of a central field. 

Then again, this term also contributes a common energy shift to all the energies that we can obtain within the single-configuration description, so this one will also be omittted. It might be useful to use this term and the previous one to estimate the energy differences between the states in different configurations, but we will not do that here.

\subsection{$\hamCoulombEESymbol$: e:e repulsion}
 
\begin{equation}
    \hamCoulombEESymbol = \hamCoulombEE = \sum_{k=0,2,4,6} F^k \hat{f}_k = \sum_{k=0,1,2,3} E_k \hat{e}^k
\end{equation} 

This term is the first we will not discard. Calculating this term for the \fn configurations was one of the contribution from Slater, as such the parameters we use to write it up are called \textit{Slater integrals}. After the analysis from Slater, Giulio Racah contributed further to the analysis of this term. The insight that Racah had was that if in a given operator one identified the parts in it that transformed nicely according to the different symmetry groups present in the problem, then calculating the necessary matrix element in all \fn configurations can   be greatly simplified.

The functions used in \ql to compute these LS-reduced matrix elements are \texttt{Electrostatic} and \texttt{fsubk}. In addition to these, the LS-reduced matrix elements of the tensor operators $\op{C}^{(k)}$ and $\op{U}^{(k)}$ are also needed. These functions are based in equations 12.16 and 12.17 from \cowan  as specialized for the case of electrons belonging to a single \fn configuration. By default this term is computed in terms of $F^k$ Slater integrals, but it can also be computed in of the $E_k$ Racah parameters, the functions \texttt{EtoF} and \texttt{FtoE} instrumental for going from one representation to the other.
 
\begin{equation}
\redbraopket{\forb^n\alpha\LSterm{2S+1}{L}}
    {\hamCoulombEESymbol}
    {\forb^n\alpha'\LSterm{2S'+1}{L'}} = \sum_{k=0,2,4,6} f_k(n,\alpha{LS},\alpha'{L'S'}) F^k 
\end{equation} 
where
\begin{multline}
    f_k(n,\alpha{LS},\alpha'{L'S'}) = \frac{1}{2} 
        \kronecker{S}{S'}
        \kronecker{L}{L'}
        \redbraopket{\forb}
            {\op{C}^{(k)}}
            {\forb}^2 \times \\
        \left\{ 
            \frac{1}{\tpo{L}} \sum_{\alpha''L''} 
                \redbraopket{\forb^n \alpha'' L'' S}
                    {\op{U}^{(k)}}
                    {\forb^n \alpha L S} 
            \redbraopket{\forb^n \alpha'' L'' S}
                {\op{U}^{(k)}} 
                {\forb^n \alpha' L S}
            - \kronecker{\alpha}{\alpha'}
                \frac{n \left(4 \forb + 2 - n\right)}
                    {(\tpo{\forb})(4 \forb + 1)} 
        \right\}
\end{multline}      


\newpage
\section{qlanth.m}

\lstinputlisting[language=Mathematica]{qlanth.m}

\section{qonstants.m} 

\lstinputlisting[language=Mathematica]{qonstants.m}

\section{qplotter.m}

\lstinputlisting[language=Mathematica]{qplotter.m}

\section{misc.m}

\lstinputlisting[language=Mathematica]{misc.m}

\end{document}
